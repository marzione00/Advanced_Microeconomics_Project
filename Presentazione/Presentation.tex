\documentclass[compress]{beamer}
\usepackage{hyperref}
\hypersetup{
    pdftitle = {Results}, % <-- Missing in PDF
    pdfauthor = {Marzio De Corato},
    pdfsubject = {Deseasoning and DID}
}

\usepackage[utf8]{inputenc}
\usepackage{graphicx}
%\author{Marzio De Corato}
\usepackage{varwidth}
\usepackage{beamerthemesplit} 

\usepackage[english]{babel}
\usepackage[backend=bibtex]{biblatex}
\addbibresource{bibliography.bib}

\useoutertheme{miniframes}


\usetheme[progressbar=frametitle]{Madrid}
\usefonttheme{professionalfonts}
\setbeamertemplate{itemize items}[triangle]
\setbeamertemplate{enumerate items}[default]
\usecolortheme{beaver}


%Information to be included in the title page:


\title[Deseasoning and DID]{}

\author{Marzio De Corato e Giulia Hadjiandrea}
\date{}

\title{Destagionalizzazione e DID}

\setbeamertemplate{enumerate items}[default]
\usecolortheme{beaver}

\begin{document}


\frame{\titlepage}

\usebackgroundtemplate{ } 

%\author{Marzio De Corato}

\section{Intro}


\begin{frame}
\frametitle{Regression}
\begin{itemize}
\item T Variabile temporale: n numero di mesi prima di Febbraio (ultimo mese senza lock-down)
\item $T1$,$T2$ e $T3$ dummy per i 3 livelli di lockdown: T1 per Marzo e Maggio (MEDIUM), T2 Aprile (HIGH) e T3 Giugno (LOW). Nei grafici questo periodo è segnato dalle barre tratteggiate.
\item $Y_{x}$ volume di vendite per il settore x 
\end{itemize}
\begin{equation}
Y_{x}-Y_{Alimentare}=\alpha+\delta_{before T1}T+\delta_{T1}T_{1}+\delta_{T2}T_{2}+\delta_{T3}T_{3}+\epsilon
\end{equation}
\end{frame}

\begin{frame}
\frametitle{Abbigliamento}
\begin{columns}[t]
\column{.5\textwidth}
\centering
\includegraphics[width=5cm,height=3.5cm]{Pic/Abbigliamento_RAW.pdf}\\
\includegraphics[width=5cm,height=4cm]{Pic/Abbigliamento.pdf}
\column{.5\textwidth}
\centering
\includegraphics[width=5cm,height=3.5cm]{Pic/Abbigliamento_Matrix.pdf}\\
\begin{itemize}
\item Rosso: Alimentari
\end{itemize}
\begin{table}[]
\begin{tabular}{l|l|l}
                & Value  & Error \\ \hline
$\delta$ Slope before T1 & -0.33  & 0.07  \\ \hline
$\delta_{T1}$   & -40.16 & 2.30  \\ \hline
$\delta_{T2}$   & -60.07 & 3.1   \\ \hline
$\delta_{T3}$   & -16.45 & 3.17 
\end{tabular}
\end{table}

\end{columns}
\end{frame}


\begin{frame}
\frametitle{Calzature}
\begin{columns}[t]
\column{.5\textwidth}
\centering
\includegraphics[width=5cm,height=3.5cm]{Pic/Calzature_RAW.pdf}\\
\includegraphics[width=5cm,height=4cm]{Pic/Calzature.pdf}
\column{.5\textwidth}
\centering
\includegraphics[width=5cm,height=3.5cm]{Pic/Calzature_Matrix.pdf}\\
\begin{itemize}
\item Rosso: Alimentari
\end{itemize}
\begin{table}[]
\begin{tabular}{l|l|l}
                & Value  & Error \\ \hline
$\delta$ Slope before T1 & -0.18  & 0.08  \\ \hline
$\delta_{T1}$   & -38.70 & 1.86   \\ \hline
$\delta_{T2}$   & -65.78 & 2.57  \\ \hline
$\delta_{T3}$   & -20.62 & 2.57 
\end{tabular}
\end{table}
\end{columns}
\end{frame}


\begin{frame}
\frametitle{Elettrodomestici}
\begin{columns}[t]
\column{.5\textwidth}
\centering
\includegraphics[width=5cm,height=3.5cm]{Pic/Elettrodomestici_RAW.pdf}\\
\includegraphics[width=5cm,height=4cm]{Pic/Elettrodomestici.pdf}
\column{.5\textwidth}
\centering
\includegraphics[width=5cm,height=3.5cm]{Pic/Elettrodomestici_Matrix.pdf}\\
\begin{itemize}
\item Rosso: Alimentari
\end{itemize}
\begin{table}[]
\begin{tabular}{l|l|l}
                & Value  & Error \\ \hline
$\delta$ Slope before T1 & -0.12  & 0.06  \\ \hline
$\delta_{T1}$   & -20.32 & 2.62   \\ \hline
$\delta_{T2}$   & -27.34 & 3.62  \\ \hline
$\delta_{T3}$   & -3.68 & 3.62
\end{tabular}
\end{table}
\end{columns}
\end{frame}


\begin{frame}
\frametitle{Fotoottica}
\begin{columns}[t]
\column{.5\textwidth}
\centering
\includegraphics[width=5cm,height=3.5cm]{Pic/Fotoottica_RAW.pdf}\\
\includegraphics[width=5cm,height=4cm]{Pic/Fotoottica.pdf}
\column{.5\textwidth}
\centering
\includegraphics[width=5cm,height=3.5cm]{Pic/Fotoottica_Matrix.pdf}\\
\begin{itemize}
\item Rosso: Alimentari
\end{itemize}
\begin{table}[]
\begin{tabular}{l|l|l}
                & Value  & Error \\ \hline
$\delta$ Slope before T1 & -0.40  & 0.07  \\ \hline
$\delta_{T1}$   & -34.14 & 2.76   \\ \hline
$\delta_{T2}$   & -52.32 & 3.82  \\ \hline
$\delta_{T3}$   & -7.86 & 3.82
\end{tabular}
\end{table}
\end{columns}
\end{frame}







%\subsection{Stars Feature}

%\begin{frame}
%\frametitle{Theoretical background - Star features }
%\begin{columns}
%\column{0.4\textwidth}
%\begin{itemize}
%\item \textbf{Main features}: For this work the main features of star are the stellar luminosity (S\_L), its temperature (S\_T) and spectral type (S\_S\_T)
%\item\textbf{H-R diagram}: with these features the Hertzsprung-Russell diagram classify the stars (the temperature and spectral type of a star are two faces of the same medal)
%\end{itemize}
%\column{0.45\textwidth}
%\includegraphics[width=\linewidth,]{Pic/S_T_T_explanation.png}
%\begin{center}
%Image taken from \cite{HR_diagram}
%\end{center}
%\end{columns}
%\end{frame}




\end{document}
