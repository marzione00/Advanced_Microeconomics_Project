%%%%%%%%%%%%%%%%%%%%%%%%%%%%%%%%%%%%%%%%%
% Arsclassica Article
% LaTeX Template
% Version 1.1 (1/8/17)
%
% This template has been downloaded from:
% http://www.LaTeXTemplates.com
%
% Original author:
% Lorenzo Pantieri (http://www.lorenzopantieri.net) with extensive modifications by:
% Vel (vel@latextemplates.com)
%
% License:
% CC BY-NC-SA 3.0 (http://creativecommons.org/licenses/by-nc-sa/3.0/)
%
%%%%%%%%%%%%%%%%%%%%%%%%%%%%%%%%%%%%%%%%%

%----------------------------------------------------------------------------------------
%	PACKAGES AND OTHER DOCUMENT CONFIGURATIONS
%----------------------------------------------------------------------------------------

\documentclass[
12pt, % Main document font size
a4paper, % Paper type, use 'letterpaper' for US Letter paper
oneside, % One page layout (no page indentation)
%twoside, % Two page layout (page indentation for binding and different headers)
headinclude,footinclude, % Extra spacing for the header and footer
BCOR5mm, % Binding correction
]{scrartcl}

\input{structure.tex} % Include the structure.tex file which specified the document structure and layout

\hyphenation{Fortran hy-phen-ation} % Specify custom hyphenation points in words with dashes where you would like hyphenation to occur, or alternatively, don't put any dashes in a word to stop hyphenation altogether

%----------------------------------------------------------------------------------------
%	TITLE AND AUTHOR(S)
%----------------------------------------------------------------------------------------

\title{\normalfont\spacedallcaps{The economic consequences of the different attitudes of a policy maker: a combined epidemiological-econometric study}} % The article title

%\subtitle{Subtitle} % Uncomment to display a subtitle

\author{\spacedlowsmallcaps{Marzio De Corato and Giulia Hadjiandrea}} % The article author(s) - author affiliations need to be specified in the AUTHOR AFFILIATIONS block

\date{} % An optional date to appear under the author(s)

%----------------------------------------------------------------------------------------

\begin{document}

%----------------------------------------------------------------------------------------
%	HEADERS
%----------------------------------------------------------------------------------------

\renewcommand{\sectionmark}[1]{\markright{\spacedlowsmallcaps{#1}}} % The header for all pages (oneside) or for even pages (twoside)
%\renewcommand{\subsectionmark}[1]{\markright{\thesubsection~#1}} % Uncomment when using the twoside option - this modifies the header on odd pages
\lehead{\mbox{\llap{\small\thepage\kern1em\color{halfgray} \vline}\color{halfgray}\hspace{0.5em}\rightmark\hfil}} % The header style

\pagestyle{scrheadings} % Enable the headers specified in this block

%----------------------------------------------------------------------------------------
%	TABLE OF CONTENTS & LISTS OF FIGURES AND TABLES
%----------------------------------------------------------------------------------------

\maketitle % Print the title/author/date block

%\setcounter{tocdepth}{2} % Set the depth of the table of contents to show sections and subsections only

%\tableofcontents % Print the table of contents

%\listoffigures % Print the list of figures%

%\listoftables % Print the list of tables

%----------------------------------------------------------------------------------------
%	ABSTRACT
%----------------------------------------------------------------------------------------

\section*{Abstract} % This section will not appear in the table of contents due to the star (\section*)

Within a standard compartmental model for the description of the dynamic of an epidemics (Susceptible-Infectious-Removed), we considered a policy-maker (PM) that impose stochastically three types of lock-down with increasing force. The probability by which the PM apply the lock-down reflects the different attitudes of a PM to face an epidemics (e.g. \textit{laissez-faire} vs very strict): thus depending on this probabilistic-parameter different scenario were simulated. For each of them we predicted a selected set of economic impacts by using parameters estimated via econometric techniques (Difference-in -Difference) on the microeconomic data of Italy. The comparison of the different impacts provide a bird's-eye view on the socio-economic consequences of the PM attitude



\newpage % Start the article content on the second page, remove this if you have a longer abstract that goes onto the second page

%----------------------------------------------------------------------------------------
%	INTRODUCTION
%----------------------------------------------------------------------------------------

\section{Introduction}

The recent pandemic due to the spread of SARS-CoV-2 virus, opened a highly debated issue about what it would be the best approach of the policy maker to face the epidemics. Differently from the pandemics of the past and in particular of the XX century (e.g. Spanish Flu, Asiatic Flu and Hong Kong Flu), for this pandemic a very large amount of data are easily accessible. As consequence not only the modelling of the virus diffusion, but also the effect on socio-economic texture for the different countries can be investigated with a finer resolution. Among the different scientific challenges that can come up in this context, a interesting one involves the effect of the attitude of the policy-maker (PM) to block the circulation of people (lock-down) in order to reduce the contagion rate (more formally the reproduction number, as described in Supporting Infomation). Indeed the policy maker can adopt, at a first approximation, a linear combination between these two extreme approaches: forcing all people to stay at home or to \textit{laissez-faire}. In the first extreme, the spread of virus is, of course, slowed down or stopped but, on the other side, the toll for such approach is that not only the economic activity (and so the income of people/firms) but also that the furniture of the primary goods are stopped. On the other side, if no lock-down measures are taken by the policy-maker, the toll to be paid will be not only the high number of deaths but also the economic damage produced by the very high number of deaths \cite{Correia2020,karlsson2014impact}. In practice the PM can adopt intermediate approaches that shut down activities that contribute much more to diffusion with respect to other (for this purpose a very fine analysis was provided by Li et al. in  \cite{li2021temporal} and by Brauner et al. in \cite{brauner2021inferring}): as consequence the lock-down efficacy, within certain limits, can be tuned. In the literature different scholars \cite{kabir2020evolutionary,rowthorn2020cost} challenged the issue of finding the optimal lock-down policy for minimizing the economic impact as well as the deaths. In particular for the model in Ref. \cite{kabir2020evolutionary} it is assumed that the policy-maker know perfectly the consequences of her choices and that she can act without delay to impose the optimal choice; finally it is assumed that the PM can impose a continuous factor for the lock-down, while, for different countries such factor seems to be much more discrete. It can be argued that most of these drawbacks of this last formidable research, are entangled with the fact that a deterministic approach was considered for the activation of the lock-downs by the policy maker. On this basis we would propose here an alternative way to model the decision of the policy maker that is based on a stochastic model instead of on a deterministic one. 

\section{Methods and models}


\section{Data description}



\section{Results and Discussion}


\bibliographystyle{unsrt}

\bibliography{sample.bib} % The file containing the bibliography

%----------------------------------------------------------------------------------------

\end{document}