%%%%%%%%%%%%%%%%%%%%%%%%%%%%%%%%%%%%%%%%%
% Arsclassica Article
% LaTeX Template
% Version 1.1 (1/8/17)
%
% This template has been downloaded from:
% http://www.LaTeXTemplates.com
%
% Original author:
% Lorenzo Pantieri (http://www.lorenzopantieri.net) with extensive modifications by:
% Vel (vel@latextemplates.com)
%
% License:
% CC BY-NC-SA 3.0 (http://creativecommons.org/licenses/by-nc-sa/3.0/)
%
%%%%%%%%%%%%%%%%%%%%%%%%%%%%%%%%%%%%%%%%%

%----------------------------------------------------------------------------------------
%	PACKAGES AND OTHER DOCUMENT CONFIGURATIONS
%----------------------------------------------------------------------------------------




\documentclass[
12pt, % Main document font size
a4paper, % Paper type, use 'letterpaper' for US Letter paper
oneside, % One page layout (no page indentation)
%twoside, % Two page layout (page indentation for binding and different headers)
headinclude,footinclude, % Extra spacing for the header and footer
BCOR5mm, % Binding correction
]{scrartcl}

\usepackage{hyperref}

\usepackage{caption}
\usepackage{subcaption}


\input{structure.tex} % Include the structure.tex file which specified the document structure and layout

\hyphenation{Fortran hy-phen-ation} % Specify custom hyphenation points in words with dashes where you would like hyphenation to occur, or alternatively, don't put any dashes in a word to stop hyphenation altogether

%----------------------------------------------------------------------------------------
%	TITLE AND AUTHOR(S)
%----------------------------------------------------------------------------------------

\title{\normalfont\spacedallcaps{The economic consequences of the different attitudes of a policy maker: a combined epidemiological-econometric study}} % The article title

%\subtitle{Subtitle} % Uncomment to display a subtitle

\author{\spacedlowsmallcaps{Marzio De Corato and Giulia Hadjiandrea}} % The article author(s) - author affiliations need to be specified in the AUTHOR AFFILIATIONS block

\date{} % An optional date to appear under the author(s)

%----------------------------------------------------------------------------------------

\begin{document}

%----------------------------------------------------------------------------------------
%	HEADERS
%----------------------------------------------------------------------------------------

\renewcommand{\sectionmark}[1]{\markright{\spacedlowsmallcaps{#1}}} % The header for all pages (oneside) or for even pages (twoside)
%\renewcommand{\subsectionmark}[1]{\markright{\thesubsection~#1}} % Uncomment when using the twoside option - this modifies the header on odd pages
\lehead{\mbox{\llap{\small\thepage\kern1em\color{halfgray} \vline}\color{halfgray}\hspace{0.5em}\rightmark\hfil}} % The header style

\pagestyle{scrheadings} % Enable the headers specified in this block

%----------------------------------------------------------------------------------------
%	TABLE OF CONTENTS & LISTS OF FIGURES AND TABLES
%----------------------------------------------------------------------------------------

\maketitle % Print the title/author/date block

%\setcounter{tocdepth}{2} % Set the depth of the table of contents to show sections and subsections only

%\tableofcontents % Print the table of contents

%\listoffigures % Print the list of figures%

%\listoftables % Print the list of tables

%----------------------------------------------------------------------------------------
%	ABSTRACT
%----------------------------------------------------------------------------------------

\section*{Abstract} % This section will not appear in the table of contents due to the star (\section*)

Within a standard compartmental model for the description of the dynamic of an epidemics (Susceptible-Infectious-Recovered-Dead), we considered a policy-maker (PM) that impose stochastically three types of lock-down with increasing force. The probability by which the PM apply the lock-down reflects the different attitudes of a PM to face an epidemics (e.g. \textit{laissez-faire} vs very strict): thus depending on this probabilistic-parameter different scenario were simulated. For each of them we predicted a selected set of economic impacts by using parameters estimated via econometric techniques (Difference-in -Difference) on the microeconomic data of Italy. The comparison of the different impacts provide a bird's-eye view on the socio-economic consequences of the PM attitude



\newpage % Start the article content on the second page, remove this if you have a longer abstract that goes onto the second page


\epigraph{"It was then that, in a moment, I saw what I must have been harboring in my hidden thoughts for a considerable time. On the one hand, Trantor possessed an extraordinarily complex social system, being a populous world made up of eight hundred smaller worlds. It was in itself a system complex enough to make psychohistory meaningful and yet it was simple enough, compared to the Empire as a whole, to make psychohistory perhaps practical"\\
I.Asimov, Prelude to foundation }


\newpage
%----------------------------------------------------------------------------------------
%	INTRODUCTION
%----------------------------------------------------------------------------------------

\section{Introduction}

The recent pandemic due to the spread of SARS-CoV-2 virus, opened a highly debated issue about what it would be the best approach of the policy maker to face the epidemics. Differently from the pandemics of the past and in particular of the XX century (e.g. Spanish Flu, Asiatic Flu and Hong Kong Flu), for this pandemic a very large amount of data are easily accessible. As consequence not only the modelling of the virus diffusion, but also the effect on socio-economic texture for the different countries can be investigated with a finer resolution. Among the different scientific challenges that can come up in this context, a interesting one involves the socio-economic effects of the attitude of the policy-maker (PM) to block the circulation of people (lock-down) in order to reduce the contagion rate (more formally the reproduction number, as described in Supporting Infomation). Indeed the policy maker can adopt, at a first approximation, a linear combination between these two extreme approaches: forcing all people to stay at home or to \textit{laissez-faire}. In the first extreme, the spread of virus is, of course, stopped but, on the other side, the toll for such approach is that not only the economic activity (and so the income of people/firms) but also that the furniture of the primary goods are stopped. On the other side, if no lock-down measures are taken by the policy-maker, the toll to be paid will be not only the high number of deaths but also the economic damage produced by the very high number of deaths \cite{Correia2020,karlsson2014impact}. In practice the PM can adopt intermediate approaches that shut down activities that contribute much more to diffusion with respect to other (for this purpose a very fine analysis was provided by Li et al. in  \cite{li2021temporal} and by Brauner et al. in \cite{brauner2021inferring}): as consequence the lock-down efficacy, within certain limits, can be tuned. In the literature different scholars \cite{kabir2020evolutionary,rowthorn2020cost} challenged the issue of finding the optimal lock-down policy for minimizing the economic impact as well as the deaths. In particular for the model in Ref. \cite{kabir2020evolutionary} it is assumed that the policy-maker know perfectly the consequences of her choices and that she can act without delay to impose the optimal choice; finally it is assumed that the PM can impose a continuous factor for the lock-down, while, for different countries such factor seems to be much more discrete. It can be argued that most of these drawbacks of this last formidable research, are entangled with the fact that a deterministic approach was considered for the activation of the lock-downs by the policy maker. On this basis we would propose here an alternative way to model the decision of the policy maker that is based on a stochastic model instead of on a deterministic one. Furthermore, differently with respect to the previous researches, that focused basically on macroeconomic impact, here we modelled the impact of the different lock-down at microeconomic level: in particular, by means of difference in difference, we evaluated the effect of the different levels of restrictions on the sales for different product types in Italy. Thus the final output of the model will be not only the cumulative deaths, but also the economic damage for each selling sector. Moreover here we also considered that there is not only an economic cost for each death, as done by \cite{kabir2020evolutionary} but there is also average cost for each infected person (referring to Italian data), because a consistent part of them may be recovered or even should take the intensive therapy. In this paper three approaches were simulated that basically shape three different PM: a very careful one, a lazy one that adopts a \textit{laissez-faire} strategy and an intermediate one. For each scenario, we will discuss the result of the simulation and then we will compare them in order to get a general insight. 

\section{Model and methods}

The model of the present study is composed by an epidemiological part, that shapes the diffusion of the virus, and then its output is used by the economic model to quantify the damage. Thus we will discuss the epidemiological part and then the economic one

\subsection{Epidemiological model}

Among the very large number of compartmental model that are available in the literature \cite{vynnycky2010introduction}, we considered as the simulator of the epidemic diffusion the simplest one: the Susceptible-Infectious-Recovered-Dead (SIRD). Our choice is motivated by the fact that this relative simple model provide the gross features of an epidemic \cite{fernandez2020estimating,al2020forecasting} with a relative small number of parameters. Thus, because the final goal of the present article is an insight on the effects of the PM care of pandemic, an not a fine grained description of the epidemics due to COVID-19, such model is suitable for our proposes\footnote{One in principle can consider a SIRD model,in which the parameters that are time-dependent, as done by Ferrari et al. in Ref. \cite{ferrari2020modelling} for the description of Italian situation. On the other side it is possible to increase the complexity of the model with other compartments as done in the following paper \cite{giordano2020modelling} by Giordano et al. Note that in this last case the resolution of 9 differential equation is required (accompanied by the estimation of a large number of parameters)}). The SIRD model,first proposed by Kermack and McKendrick \cite{kermack1927contribution}, is given by the following set of differential equations\cite{vynnycky2010introduction}:  
\begin{equation} 
\begin{split}
\dfrac{dS(t)}{dt} & = - \frac{\beta I(t) S(t)}{N} \\
\dfrac{dI(t)}{dt} & = \frac{\beta I(t) S(t)}{N} -\gamma I(t) -\mu I(t) \\
\dfrac{dR(t)}{dt} & = \gamma I(t) \\
\dfrac{dD(t)}{dt} & = \mu I(t)
\end{split}
\end{equation}
where S is the number of people that are still susceptible, I the number of people that are infected  and R people that are recovered while D are people that are death. N denotes the total popolation, that for the timing considered here can be considered fixed \footnote{Othervise, if longer horizontal timing is considered, it is necessary to consider also a source term for the births and a well term for the natural deaths. For further details see \cite{vynnycky2010introduction}}. On the other side $\beta$,$\gamma$ and $\mu$ are the parameters that shape the probability by which one individual in the model moves from a compartment to another: in particular $\beta$ is the probability to be infected, $\gamma $ the probability to recover and $\mu$ the probability to die (basically the lethality defined  as the probability to die conditionally to be ill). Usually epidemiologist are interested in the ratio:  
\begin{equation} 
R_{0}=\dfrac{\beta}{\gamma}
\end{equation}
know as the basic reproduction factor. This number, that is the average number of people that are infected by a single individual, describe if the epidemic is in a negative feedback (R$_{0}$<1), stationary (R$_{0}$=1), or in a positive feedback (R$_{0}$>1). As consequence if the epidemic is in a negative feedback it will be going to dissipate, if it is in positive feedback it will grow. Note that in this simple model since the parameters are not time dependent this factor is constant. If, as performed by Ferrari et al. \cite{ferrari2020modelling}, time dependent parameters are considered a time depended parameter called $R_{t}$. For the present work, we limited to constant parameters: in particular we considered the parameter estimation for Lombardy provided by Neves and Guerrero in Ref \cite{neves2020predicting}: $\beta$ was set equal to 0.55 while $\gamma$ equal to $\frac{1}{7}$. The overall population $N$ was set to 60M, in order to simulate Italian population.  Within the daily temporal evolution of this model, that was obtained by numerically solving the differential equation above via the \textit{DeSolve} package, we considered a trigger activated by the PM every 7 days: if the number of infected people normalized by the overall population is more than $1\times 10^{-7}$, there is a probability that the PM impose laws that multiply the $\beta$ factor by $0.7$, if the normalized infected people are more than 10 the previous threshold  she will impose, with a certain probability, restrictions that multiply the $\beta$ starting $\beta$ factor by $0.25$, finally if the  threshold is exceeded more than 50 times the PM will impose with a certain probability restrictive measures that multiply the $\beta$ by $0.025$. These attenuation parameters were adjusted taking in consideration the results of Marziano et al. in Ref. \cite{marziano2021retrospective}. As we said the PM act with a certain probability, more formally stochastically: each week a random number (from zero to one) is extracted, if this is higher than a certain threshold, the relative restrictive decision is taken, otherwise not. The threshold value capture the PM attitude to impose the lock-down: lower values model a careful PM, hight value a lazy one. In this way the model is able to simulate different scenarios for the different PM attitude: as we will see this can produce two very different results. At the end of the simulation, beside the values given by the standard SIRD model (recovered and deaths), the model also provide the number of days in which each restriction was active: these values are then used for the economic model in order to evaluate the economic effect due to the restrictions and PM strategy. 

\subsection{Economic model}

The economic impact for each epidemiological scenario is shaped as follow: a first set of parameters, as the economic value of a death and of being infected by COVID-19 ,was taken directly from the reports/documentation of official sources; other parameters as the effect on selling for different areas and on the unemployment benefit (Cassa Integrazione Guadagni) were evaluated with empirical approaches from raw data. Concerning the first set, the number of deaths is multiplied by the maximum compensation value provided by the Court of Milan \cite{tribunaleMilano} for manslaughter (300k EUR). This choice is based on the idea that, if the PM act improperly, can be incriminated for manslaughter (with the consent of parliament that has to validate the incrimination) and than, if judged guilty, charged by this amount for each death\footnote{In principle the judge also keep into account the age of the deaths: this in principle require a model in which also the age of people is keep into account. In this case however a system of partial differential equation should be solved making the calculation and the computational cost incredibly high.}. Beside this impact there is also the cost associated to the medical care of each ill people, for this we considered the average value calculated by National Anti-Corruption Authority (ANAC) \cite{Anac}: 28.180 kEUR \footnote{It is worth noting that, in principle, there is also another import health-care impact due to the fact that the ill people for COVID-19 saturate the health system thus making it unreachable for other disease. This spillover translates in to more death and more ill people with respect to the baseline situation where there isn't an pandemic: however, by now, this effect is difficult to quantify and so we did not included in to the present model}. Among the different sectors affected by the pandemic and the consequent lock-down, we focused on the selling for the following ATECO-2007 \cite{Ateco2007} categories\footnote{The category Equipment for information technology/telephony and telecommunications was omitted since there is no point to argue that the lock-down affected, in reverse there it is more likely to argue that is increased due to the fact that people was forced to work remotely \cite{guo2020digitalization}}  \footnote{In the rest part of the paper these categories will be referred as the part of the name labelled in blue. 
}: \textcolor{blue}{Food}, \textcolor{blue}{Clothing} and furs, \textcolor{blue}{Footwear}/leather and travel articles, \textcolor{blue}{Furniture}/textile articles/furnishings for the home, 
\textcolor{blue}{Appliances}/radios/televisions and tape recorders, \textcolor{blue}{Photo-optics}/films/compact discs/ audio-video cassettes and musical instruments, Durable and non-durable \textcolor{blue}{Household kinds}, \textcolor{blue}{Household tools} and hardware,  \textcolor{blue}{Games}/toys/sports and camping articles. The choice to use the selling as a parameter for the evaluation of the lock-down lies on the fact that with them it is possible to capture not only the contraction for each sector but also the loss for the public treasury due to to the reduced incomes from the VAT\footnote{For this purpose another sector that in principle can be also considered is the contraction of fuel selling, due to the reduced mobility, where in addition to the VAT there is also a fixed taxation (accisa). Such calculation may be considered as a future outlook of this work}. Beside the selling we also considered the unemployment benefit for the following ATECO sectors: \textcolor{blue}{Manufacturing} activities, \textcolor{blue}{Construction} , \textcolor{blue}{Wholesale} and retail trade/ repair of motor vehicles-motorcycles and personal and household goods, \textcolor{blue}{Real estate} activities, rental, IT, research, business services\footnote{On contrary with respect to the category Equipment for information technology of selling, here the IT contribution is overwhelmed by the other ones (e.g Real Estate), making it negligible. Thus, in this case we included the whole sector}. In this case the choice to use also this parameter is based on the fact that this is the first aid provided by the government for the firms that were damaged by the contraction in demand as well as by the fact that for different sector of them the production was also banned by law during the lock-down \footnote{In principle one can be also interested to disentangle these two effects: in this case an interesting option is to study the second-wave of the epidemics in Italy that happened in the second half of the 2020. Contrary with respect to the first wave, the lock-down did not banned by law the production. As told previously this option was not considered by the authors since the most of the economic data were available, with the temporal resolution considered in this work, only at national level.}. For the empirical evaluation of the impact on selling and unemployment benefit we performed a Difference-In-Difference (DID) where multiple times where considered as used in Refs. \cite{draca2011panic} and \cite{wooldridge2012introductory}. The following regression was performed: 
\begin{equation}
Y_{outcome}=\alpha+\sum_{i=1}^{3}\beta_{i}T_{i}+\sum_{i=1}^{3}\delta_{i}(C \cdot T_{i})+\epsilon
\label{DID-eq}
\end{equation}
where $Y$ is the selcted outcome, $\alpha$ the intercept, $T_{i}$ a dummy variable for the lock-down timing i,  C a dummy for the treated group and $\epsilon$ a error term. [...].  The DID considered  here has three different times for the lock-down:  for the sales we considered, as  done for the epidemiological model,  the months of March 2020 and May 2020 as medium lock-down, the month of April 2020 as High lock-down and June 2020 as weak lock-down.  On the other side  concerning the subsidies we considered as medium lock-down only May 2020: such choice was motivated by the fact that the effect of firms to use the subsidies for workers was slightly delayed with respect to effect lock-down effects on sales.  Performing the DID we estimated the coefficients for the effect of each intensity of lock-down on sales and subsidies,  then we we rescaled them for a week (multiplying them by a factor of $\frac{7}{30})$. Finally we multiplied the number obtained from the previous calculation by the number of lock-down weeks, with the respective intensity,  in each scenario simulated via the epidemiological model described before.  As consequence we obtained,  for each scenario, an economics simulation based on parameters obtained from an empirical evaluation. 



\section{Data description}

The monthly sales data were take retrieved from the website of National Institute of Statistics \cite{istat}.  In particular we considered the period starting from 2018 up to June 2020. The choice to not consider the month after June and in particular the last part of the year lies on the fact that in the latter the lock-down were imposed at regional level and not to national level \footnote{With the only exception of Christmas holiday}  \footnote{In principle one may ask why the present analysis was not performed on regional cluster making it more flexible: unfortunately the economic data used here,  
as far as the author knows, were not available,  at all for regional cluster. Moreover if regional cluster were considered it was necessary to model an ensemble of SIRD model that communicate each other with a defined rate (that change also with respect to the lock-down restriction).  This make the model much more complicated.  However if all the data that are necessary for performing the analysis will become availbale the author may consider, as an outlook, to extend the present analysis to regional clusters} As discussed in Supporting Information,  these data were not de-seasoned: thus we performed a de-seasoning via the \textit{Forecast} R package \cite{forecast} that uses a Hilbert-Huang transform  \cite{huang1998empirical} for the decomposition of a time series data.  Concerning the subsidies we retried the data from the \textit{Osservatorio Cassa integrazione guadagni e fondi di solidarietà} on the \textit{Istituto nazionale della previdenza sociale} webpage \cite{inps}.  It is worth nothing that here we considered only the authorized subsidy and not the asked one.  Furthermore these data,  differently with respect to the sales one were not affected by seasonality noise,  and thus no de-seasoning was necessary. 


\section{Results and Discussion}

As done for the section Model and methods we will divide the discussion of the result in the following way: first the result of the economic model will be presented,  then on the basis of them we will discuss the  scenarios obtained with them via the epidemiological model.  Finally we will discuss the overall results. 


\paragraph{DID}

For each sales category we run the regression reported in the Eq.  \ref{DID-eq} on de-seasoned data reported in Fig \ref{Selling_plots} and \ref{Selling_plots_II}.  The coefficients obtained are reported in Tab.  \ref{selling_DID_table_delta}.  As one can point out from the plots,  it is true that there is a pre-trend in the data before the event,  however as proved numerically in the Supporting Info,  the slope of this pre-trend is almost two order of magnitude lesser with respect to the slope in the lock-down T$_{1}$ and $T_{2}$.  As consequence this pre-trend,  compared to the lock-down effect,  can be considered negligible. 


\paragraph{Scenarios}


\paragraph{Conclusions}

\begin{figure}
     \centering
     \begin{subfigure}[b]{0.45\textwidth}
         \centering
         \includegraphics[width=\textwidth]{Figures/Abbigliamento.pdf}
         \caption{ \textcolor{blue}{Clothing} (blue) }
         \label{fig:y equals x}
     \end{subfigure}
     \hfill
     \begin{subfigure}[b]{0.45\textwidth}
         \centering
         \includegraphics[width=\textwidth]{Figures/Calzature.pdf}}
         \caption{\textcolor{OliveGreen}{Footwear} (green)}
         \label{fig:three sin x}
     \end{subfigure}
          \begin{subfigure}[b]{0.45\textwidth}
         \centering
         \includegraphics[width=\textwidth]{Figures/Elettrodomestici.pdf}
         \caption{ \textcolor{Orange}{Appliances} (orange)}
         \label{fig:y equals x}
     \end{subfigure}
     \hfill
     \begin{subfigure}[b]{0.45\textwidth}
         \centering
         \includegraphics[width=\textwidth]{Figures/Fotoottica.pdf}}
          \caption{\textcolor{Blue}{Photo-optics} (blue)}
         \label{fig:three sin x}
     \end{subfigure}
\caption{Selling data (I), with baseline of 2015,  as provided by \cite{istat} de-seasoned via Forecast package \cite{forecast} for the categories  analysed in this paper compared with \textcolor{red}{food} (red) category.  The timing for each lock-down is marked with a dashed line: red for high, orange medium and green for low.  Note that despite there is a pre-trend this is negligible with respect to the slope of medium and high lock-down slopes}
\label{Selling_plots}
\end{figure}



\begin{figure}
     \centering
     \hfill
     \begin{subfigure}[b]{0.45\textwidth}
         \centering
         \includegraphics[width=\textwidth]{Figures/Mobili.pdf}}
         \caption{\textcolor{OliveGreen}{Household kids} (Orchid)}
         \label{fig:three sin x}
     \end{subfigure}
        \begin{subfigure}[b]{0.45\textwidth}
         \centering
         \includegraphics[width=\textwidth]{Figures/Utilenseria.pdf}
         \caption{ \textcolor{Orange}{Household tools} (Light green)}
         \label{fig:y equals x}
     \end{subfigure}
             \begin{subfigure}[b]{0.45\textwidth}
         \centering
         \includegraphics[width=\textwidth]{Figures/Giocattoli.pdf}
         \caption{ \textcolor{Thistle}{Games} (Thistle)}
         \label{fig:y equals x}
     \end{subfigure}
\caption{Selling data (II), with baseline of 2015,  as provided by \cite{istat} de-seasoned via Forecast package \cite{forecast} for the categories  analysed in this paper compared with \textcolor{red}{food} (red) category.  The timing for each lock-down is marked with a dashed line: red for high, orange medium and green for low.  Note that despite there is a pre-trend this is negligible with respect to the slope of medium and high lock-down slopes}
\label{Selling_plots_II}
\end{figure}

\begin{landscape}
\begin{table}[]
\caption{
\begin{flushleft}
$\delta$ cofficients as obtained by the DID  regression, for the selling data de-seasoned,  according to equation \ref{DID-eq} \\ for the different lock-down timings. The values of the  intercept ($\alpha$), C and T$_{i}$ are provided in the Supporting Info. 
\end{flushleft}
}
\begin{tabular}{l|l|l|l|l|l|l|l|l|l}
                                       & $\delta_{1} $&   $ \sigma_{\delta_{1}}$   & t      & $\delta_{2} $ & $\sigma_{\delta_{2}$ & t      &  $\delta_{3}$ & $\sigma_{\delta_{3}$  & t     \\ \hline
{\color[HTML]{3531FF} Clothing}        & -60.07 & 2.77  & -21.64 & 40.16  & 2.01  & -19.99 & -16.45 & 2.77  & -5.92 \\ \hline
{\color[HTML]{009901} Footwear}        & -65.78 & 2.39  & -27.49 & -20.62 & 1.73  & -22.35 & -20.62 & 2.39  & -8.62 \\ \hline
{\color[HTML]{F56B00} Appliances}      & -27.34 & 3.45  & -7.90  & -20.32 & 2.50  & -8.11  & -3.68  & 3.45  & 1.06  \\ \hline
{\color[HTML]{00009B} Photo-optics}    & -52.32 & 3.52  & -14.84 & -34.14 & 2.55  & -13.38 & -7.86  & 3.52  & 2.23  \\ \hline
{\color[HTML]{6200C9} Household kids}  & -18.99 & 2.07  & -9.13  & -10.51 & 1.50  & -6.98  & -3.61  & 2.07  & -1.74 \\ \hline
{\color[HTML]{00D2CB} Household tools} & -30.23 & 4.85  & -6.23  & -13.02 & 3.51  & -3.70  & 2.33   & 4.85  & 0.48  \\ \hline
{\color[HTML]{FFC9FD} Games}           & -30.23 & 4.85  & -6.22  & -13.02 & 3.51  & -3.70  & 2.33   & 4.85  & 0.48 

\end{tabular}
\label{selling_DID_table_delta}
\end{table}
\end{landscape}
\newpage


\bibliographystyle{unsrt}

\bibliography{sample.bib} % The file containing the bibliography

%----------------------------------------------------------------------------------------

\end{document}